% Options for packages loaded elsewhere
% Options for packages loaded elsewhere
\PassOptionsToPackage{unicode}{hyperref}
\PassOptionsToPackage{hyphens}{url}
\PassOptionsToPackage{dvipsnames,svgnames,x11names}{xcolor}
%
\documentclass[
  a4paper,
  12pt,
  english,
  toctotoc]{ManitSynopsis}
\usepackage{xcolor}
\usepackage[a4paper,inner=25mm,outer=25mm,bindingoffset=5mm,top=15mm,bottom=15mm]{geometry}
\usepackage{amsmath,amssymb}
\setcounter{secnumdepth}{5}
\usepackage{iftex}
\ifPDFTeX
  \usepackage[T1]{fontenc}
  \usepackage[utf8]{inputenc}
  \usepackage{textcomp} % provide euro and other symbols
\else % if luatex or xetex
  \usepackage{unicode-math} % this also loads fontspec
  \defaultfontfeatures{Scale=MatchLowercase}
  \defaultfontfeatures[\rmfamily]{Ligatures=TeX,Scale=1}
\fi
\usepackage{lmodern}
\ifPDFTeX\else
  % xetex/luatex font selection
\fi
% Use upquote if available, for straight quotes in verbatim environments
\IfFileExists{upquote.sty}{\usepackage{upquote}}{}
\IfFileExists{microtype.sty}{% use microtype if available
  \usepackage[]{microtype}
  \UseMicrotypeSet[protrusion]{basicmath} % disable protrusion for tt fonts
}{}
\makeatletter
\@ifundefined{KOMAClassName}{% if non-KOMA class
  \IfFileExists{parskip.sty}{%
    \usepackage{parskip}
  }{% else
    \setlength{\parindent}{0pt}
    \setlength{\parskip}{6pt plus 2pt minus 1pt}}
}{% if KOMA class
  \KOMAoptions{parskip=half}}
\makeatother
% Make \paragraph and \subparagraph free-standing
\makeatletter
\ifx\paragraph\undefined\else
  \let\oldparagraph\paragraph
  \renewcommand{\paragraph}{
    \@ifstar
      \xxxParagraphStar
      \xxxParagraphNoStar
  }
  \newcommand{\xxxParagraphStar}[1]{\oldparagraph*{#1}\mbox{}}
  \newcommand{\xxxParagraphNoStar}[1]{\oldparagraph{#1}\mbox{}}
\fi
\ifx\subparagraph\undefined\else
  \let\oldsubparagraph\subparagraph
  \renewcommand{\subparagraph}{
    \@ifstar
      \xxxSubParagraphStar
      \xxxSubParagraphNoStar
  }
  \newcommand{\xxxSubParagraphStar}[1]{\oldsubparagraph*{#1}\mbox{}}
  \newcommand{\xxxSubParagraphNoStar}[1]{\oldsubparagraph{#1}\mbox{}}
\fi
\makeatother

\usepackage{color}
\usepackage{fancyvrb}
\newcommand{\VerbBar}{|}
\newcommand{\VERB}{\Verb[commandchars=\\\{\}]}
\DefineVerbatimEnvironment{Highlighting}{Verbatim}{commandchars=\\\{\}}
% Add ',fontsize=\small' for more characters per line
\usepackage{framed}
\definecolor{shadecolor}{RGB}{241,243,245}
\newenvironment{Shaded}{\begin{snugshade}}{\end{snugshade}}
\newcommand{\AlertTok}[1]{\textcolor[rgb]{0.68,0.00,0.00}{#1}}
\newcommand{\AnnotationTok}[1]{\textcolor[rgb]{0.37,0.37,0.37}{#1}}
\newcommand{\AttributeTok}[1]{\textcolor[rgb]{0.40,0.45,0.13}{#1}}
\newcommand{\BaseNTok}[1]{\textcolor[rgb]{0.68,0.00,0.00}{#1}}
\newcommand{\BuiltInTok}[1]{\textcolor[rgb]{0.00,0.23,0.31}{#1}}
\newcommand{\CharTok}[1]{\textcolor[rgb]{0.13,0.47,0.30}{#1}}
\newcommand{\CommentTok}[1]{\textcolor[rgb]{0.37,0.37,0.37}{#1}}
\newcommand{\CommentVarTok}[1]{\textcolor[rgb]{0.37,0.37,0.37}{\textit{#1}}}
\newcommand{\ConstantTok}[1]{\textcolor[rgb]{0.56,0.35,0.01}{#1}}
\newcommand{\ControlFlowTok}[1]{\textcolor[rgb]{0.00,0.23,0.31}{\textbf{#1}}}
\newcommand{\DataTypeTok}[1]{\textcolor[rgb]{0.68,0.00,0.00}{#1}}
\newcommand{\DecValTok}[1]{\textcolor[rgb]{0.68,0.00,0.00}{#1}}
\newcommand{\DocumentationTok}[1]{\textcolor[rgb]{0.37,0.37,0.37}{\textit{#1}}}
\newcommand{\ErrorTok}[1]{\textcolor[rgb]{0.68,0.00,0.00}{#1}}
\newcommand{\ExtensionTok}[1]{\textcolor[rgb]{0.00,0.23,0.31}{#1}}
\newcommand{\FloatTok}[1]{\textcolor[rgb]{0.68,0.00,0.00}{#1}}
\newcommand{\FunctionTok}[1]{\textcolor[rgb]{0.28,0.35,0.67}{#1}}
\newcommand{\ImportTok}[1]{\textcolor[rgb]{0.00,0.46,0.62}{#1}}
\newcommand{\InformationTok}[1]{\textcolor[rgb]{0.37,0.37,0.37}{#1}}
\newcommand{\KeywordTok}[1]{\textcolor[rgb]{0.00,0.23,0.31}{\textbf{#1}}}
\newcommand{\NormalTok}[1]{\textcolor[rgb]{0.00,0.23,0.31}{#1}}
\newcommand{\OperatorTok}[1]{\textcolor[rgb]{0.37,0.37,0.37}{#1}}
\newcommand{\OtherTok}[1]{\textcolor[rgb]{0.00,0.23,0.31}{#1}}
\newcommand{\PreprocessorTok}[1]{\textcolor[rgb]{0.68,0.00,0.00}{#1}}
\newcommand{\RegionMarkerTok}[1]{\textcolor[rgb]{0.00,0.23,0.31}{#1}}
\newcommand{\SpecialCharTok}[1]{\textcolor[rgb]{0.37,0.37,0.37}{#1}}
\newcommand{\SpecialStringTok}[1]{\textcolor[rgb]{0.13,0.47,0.30}{#1}}
\newcommand{\StringTok}[1]{\textcolor[rgb]{0.13,0.47,0.30}{#1}}
\newcommand{\VariableTok}[1]{\textcolor[rgb]{0.07,0.07,0.07}{#1}}
\newcommand{\VerbatimStringTok}[1]{\textcolor[rgb]{0.13,0.47,0.30}{#1}}
\newcommand{\WarningTok}[1]{\textcolor[rgb]{0.37,0.37,0.37}{\textit{#1}}}

\usepackage{longtable,booktabs,array}
\usepackage{calc} % for calculating minipage widths
% Correct order of tables after \paragraph or \subparagraph
\usepackage{etoolbox}
\makeatletter
\patchcmd\longtable{\par}{\if@noskipsec\mbox{}\fi\par}{}{}
\makeatother
% Allow footnotes in longtable head/foot
\IfFileExists{footnotehyper.sty}{\usepackage{footnotehyper}}{\usepackage{footnote}}
\makesavenoteenv{longtable}
\usepackage{graphicx}
\makeatletter
\newsavebox\pandoc@box
\newcommand*\pandocbounded[1]{% scales image to fit in text height/width
  \sbox\pandoc@box{#1}%
  \Gscale@div\@tempa{\textheight}{\dimexpr\ht\pandoc@box+\dp\pandoc@box\relax}%
  \Gscale@div\@tempb{\linewidth}{\wd\pandoc@box}%
  \ifdim\@tempb\p@<\@tempa\p@\let\@tempa\@tempb\fi% select the smaller of both
  \ifdim\@tempa\p@<\p@\scalebox{\@tempa}{\usebox\pandoc@box}%
  \else\usebox{\pandoc@box}%
  \fi%
}
% Set default figure placement to htbp
\def\fps@figure{htbp}
\makeatother

\ifLuaTeX
  \usepackage{luacolor}
  \usepackage[soul]{lua-ul}
\else
  \usepackage{soul}
\fi




\setlength{\emergencystretch}{3em} % prevent overfull lines

\providecommand{\tightlist}{%
  \setlength{\itemsep}{0pt}\setlength{\parskip}{0pt}}



 
\usepackage[backend=biber,style=ieee]{biblatex}
\addbibresource{references.bib}


% \usepackage[utf8]{inputenc} % Required for inputting international characters
%\usepackage[T1]{fontenc} % Output font encoding for international characters; causes problems for xelatex

%\usepackage{mathpazo} % Use the Palatino font by default

% \usepackage[backend=bibtex, style=authoryear, natbib=true]{biblatex} % Use the bibtex backend with the authoryear citation style (which resembles APA)
% \usepackage{biblatex}

\usepackage[autostyle=true]{csquotes} % Required to generate language-dependent quotes in the bibliography
\usepackage{scrlayer-scrpage}
% \usepackage[linesnumbered,ruled,vlined]{algorithm2e}
\usepackage{algpseudocode}

\usepackage[none]{hyphenat} % Prevent LaTeX from hyphenating words
\sloppy                      % Loosen spacing to avoid overfull lines
\setlength{\emergencystretch}{3em} % Extra flexibility for justification

%----------------------------------------------------------------------------------------
%	MARGINS
%----------------------------------------------------------------------------------------

\geometry{
	headheight=2ex,
	includehead,
	includefoot
}

\raggedbottom

\AtBeginDocument{
\hypersetup{pdftitle=\ttitle} % Set the PDF's title to your title
\hypersetup{pdfauthor=\authorName} % Set the PDF's author to your name
\hypersetup{pdfkeywords=\keywordnames} % Set the PDF's keywords to your keywords
}


% ---- Chapter title settings
\booltrue{chapteroneline}
\renewcommand{\mdtChapapp}{}
\renewcommand{\autodot}{}
\renewcommand{\abovechapterskip}{\vspace*{0pt}}

% Remove "Contents" from header after Table of Contents
\patchcmd{\tableofcontents}{\@starttoc{toc}}{%
  \markboth{}{}%
  \@starttoc{toc}%
}{}{}

% Reduce font size and spacing in bibliography
\AtBeginDocument{%
  \AtBeginBibliography{%
    \footnotesize
    \setlength{\bibitemsep}{0pt}
    \setlength{\parskip}{0pt}
  }%
}


\renewcommand{\algorithmicrequire}{\textbf{Input:}}
\renewcommand{\algorithmicensure}{\textbf{Output:}}
\makeatletter
\@ifpackageloaded{bookmark}{}{\usepackage{bookmark}}
\makeatother
\makeatletter
\@ifpackageloaded{caption}{}{\usepackage{caption}}
\AtBeginDocument{%
\ifdefined\contentsname
  \renewcommand*\contentsname{Table of contents}
\else
  \newcommand\contentsname{Table of contents}
\fi
\ifdefined\listfigurename
  \renewcommand*\listfigurename{List of Figures}
\else
  \newcommand\listfigurename{List of Figures}
\fi
\ifdefined\listtablename
  \renewcommand*\listtablename{List of Tables}
\else
  \newcommand\listtablename{List of Tables}
\fi
\ifdefined\figurename
  \renewcommand*\figurename{Figure}
\else
  \newcommand\figurename{Figure}
\fi
\ifdefined\tablename
  \renewcommand*\tablename{Table}
\else
  \newcommand\tablename{Table}
\fi
}
\@ifpackageloaded{float}{}{\usepackage{float}}
\floatstyle{ruled}
\@ifundefined{c@chapter}{\newfloat{codelisting}{h}{lop}}{\newfloat{codelisting}{h}{lop}[chapter]}
\floatname{codelisting}{Listing}
\newcommand*\listoflistings{\listof{codelisting}{List of Listings}}
\makeatother
\makeatletter
\makeatother
\makeatletter
\@ifpackageloaded{caption}{}{\usepackage{caption}}
\@ifpackageloaded{subcaption}{}{\usepackage{subcaption}}
\makeatother
\makeatletter
\@ifpackageloaded{algorithm}{}{\usepackage{algorithm}}
\makeatother
\makeatletter
\@ifpackageloaded{algpseudocode}{}{\usepackage{algpseudocode}}
\makeatother
\makeatletter
\@ifpackageloaded{caption}{}{\usepackage{caption}}
\makeatother
\usepackage{bookmark}
\IfFileExists{xurl.sty}{\usepackage{xurl}}{} % add URL line breaks if available
\urlstyle{same}
\hypersetup{
  pdftitle={Your Thesis Title Here},
  pdfauthor={Your Name},
  colorlinks=true,
  linkcolor={blue},
  filecolor={Maroon},
  citecolor={blue},
  urlcolor={red},
  pdfcreator={LaTeX via pandoc}}


% -------------------------------------------------------
% Thesis Title & Author Information
% -------------------------------------------------------
\thesistitle{Your Thesis Title
Here} % Thesis title, used in the title and abstract, print it elsewhere with \ttitle
\author{Your
Name} % Author's name, used in the title page and abstract, print it elsewhere with \authorName
\scholarid{YOUR\_SCHOLAR\_NUMBER} % Scholar ID
\phdstart{Month Year} % PhD start date
\phdend{Month Year} % PhD end date
\institute{Maulana Azad National Institute of Technology, Bhopal}

% -------------------------------------------------------
% Degree and Examiner Information
% -------------------------------------------------------
\degree{Doctor of Philosophy} % Degree name (e.g., Doctor of Philosophy)
\examiner{} % Examiner's name, currently not used in the template, print it elsewhere with \examinerName

% -------------------------------------------------------
% University Information
% -------------------------------------------------------
\university{}

\universitylineone{Maulana Azad} % University name first line
\universitylinetwo{National Institute of
Technology,} % University name second line
\universitylocation{Bhopal - 462003 (INDIA)} % University location

% -------------------------------------------------------
% Department & Faculty Information
% -------------------------------------------------------
\department{Your Department Name} % Thesis department name

\faculty{}

\group{}

% -------------------------------------------------------
% Research Subject and Keywords
% -------------------------------------------------------
\subject{} % Research subject (not currently used)
% \keywords{Your Keywords Here} % Thesis keywords
% -------------------------------------------------------
% Supervisor Information (Primary Supervisor)
% -------------------------------------------------------
\supervisor{Dr.~Supervisor Name} % Supervisor's full name
\supdesignation{Designation} % Supervisor's designation (e.g., Associate Professor)
\supdepartment{Your Department Name} % Supervisor's department
\supinstitute{Maulana Azad National Institute of Technology,
Bhopal} % Supervisor's institute/university

% -------------------------------------------------------
% Co-Supervisor Information (Secondary Supervisor)
% -------------------------------------------------------
\cosupervisor{Dr.~Co-Supervisor Name} % Co-Supervisor's full name
\cosupdesignation{Designation} % Co-Supervisor's designation
\cosupdepartment{Department Name} % Co-Supervisor's department
\cosupinstitute{Institute Name} % Co-Supervisor's institute/university

% -------------------------------------------------------
% Table of Contents Depth
% -------------------------------------------------------
\setcounter{tocdepth}{1} % Controls depth of sections in ToC
\begin{document}
\frontmatter % Use roman page numbering style (i, ii, iii, iv...) for the pre-content pages

\pagestyle{plain} % Default to the plain heading style until the thesis style is called for the body content

%----------------------------------------------------------------------------------------
%	TITLE PAGE
%----------------------------------------------------------------------------------------

\begin{titlepage}
\thispagestyle{empty} % No header/footer
\begin{center}
% Add flexible spacing above
\vfill

% Thesis Title in uppercase
{\fontsize{16pt}{20pt}\selectfont \bfseries \MakeUppercase{\ttitle}\par}\vspace{6mm}

\textbf{PRE-THESIS SYNOPSIS}\\% Thesis type

\vspace{3mm} 

\textbf{\textit{submitted in partial fulfillment of the\\ requirements for the award of the degree}}

\vspace{1mm} 

\textit{of} \vspace{1mm} % "of" with balanced spacing

\MakeUppercase{\textbf{\degreeName}} \vspace{1mm} % Degree name in uppercase

\textit{in} \vspace{1mm} % "in" and Department name

\MakeUppercase{\textbf{\Department}} \vspace{1mm}

\textit{by} \vspace{1mm} % "by" and Author details

\MakeUppercase{\textbf{\authorName}} \\[1mm] % Author without link

\textbf{(Sch. No.: \scholarID)}

\textbf{Under the Supervision}

\textit{of} \vspace{2mm}

\textbf{\supervisorName} \& \textbf{\cosupervisorName}


\vfill

% Logo
\includegraphics[height=1.75in]{images/logo.png}

\vfill

{\fontsize{14pt}{16pt}\selectfont\scshape % University details in uppercase
\MakeUppercase{\textbf{\universityLineOne}} \\ 
\MakeUppercase{\textbf{\universityLineTwo}} \\ 
\MakeUppercase{\textbf{\universityLocation}}\par} 
\vspace{2mm}

{\fontsize{14pt}{16pt}\selectfont \MakeUppercase{\textbf{Month,
Year}}} % Date

\vspace{3mm}
{\small\textbf{Proud to be part of An Institute of National Importance}}
\end{center}
\end{titlepage}

% \cleardoublepage % Ensures next section starts on the right side
% 
% %----------------------------------------------------------------------------------------
% %	COPYRIGHT PAGE
% %----------------------------------------------------------------------------------------
% \newpairofpagestyles{copyrightpage}{% Define a new page style
%     \clearpairofpagestyles % Clear default header/footer
%     \cfoot{\textbf{Proud to be part of An Institute of National Importance}} % Centered in footer
%     \ofoot{\pagemark} % Page number at the bottom-right
% }
% 
% \begin{center}
%     \vspace*{0.3\textheight}
%     \textcopyright\ {\MakeUppercase{\textbf{\universityLineOne}} \MakeUppercase{\textbf{\universityLineTwo}} \\ 
%     \MakeUppercase{\textbf{\universityLocation}}\par} 
%     \textbf{ALL RIGHTS RESERVED}
%     \vfill % Pushes the next text to the bottom
%     \textbf{Proud to be part of An Institute of National Importance}
% \end{center}

\cleardoublepage
% 
% %----------------------------------------------------------------------------------------
% %	DECLARATION PAGE
% %----------------------------------------------------------------------------------------
% 
% \addchaptertocentry{\authorshipname} 
% 
% 


% 
%----------------------------------------------------------------------------------------
%	CONTENTS, FIGURES, AND TABLES
%----------------------------------------------------------------------------------------

\begingroup
\hypersetup{linkcolor=black}

\tableofcontents
% \listoffigures
% \listoftables

\endgroup

%----------------------------------------------------------------------------------------
%	ABBREVIATIONS, SYMBOLS, AND DEDICATION SECTIONS
%----------------------------------------------------------------------------------------

% % 
% % 
% % 
% 
\cleardoublepage % Ensures Chapter 1 starts on the right page in two-sided layout

%----------------------------------------------------------------------------------------
%	MAIN THESIS CONTENT
%----------------------------------------------------------------------------------------

\mainmatter
\pagestyle{plain}  
% Define some commands to keep the formatting separated from the content 
\newcommand{\keyword}[1]{\textbf{#1}}
\newcommand{\tabhead}[1]{\textbf{#1}}
\newcommand{\code}[1]{\texttt{#1}}
\newcommand{\file}[1]{\texttt{\bfseries#1}}
\newcommand{\option}[1]{\texttt{\itshape#1}}

\floatname{algorithm}{Algorithm}

\numberwithin{algorithm}{chapter}


\bookmarksetup{startatroot}

\chapter{Introduction}\label{sec-introduction}

This template demonstrates how to write a thesis synopsis using Quarto.
Each section below shows the syntax and rendered output.

\section{Sections and Subsections}\label{sec-sections}

Use \texttt{\#} symbols for headings. Each \texttt{\#} represents a
level:

\begin{Shaded}
\begin{Highlighting}[]
\FunctionTok{\# Chapter Title (appears in Table of Contents)}
\FunctionTok{\#\# Section}
\FunctionTok{\#\#\# Subsection  }
\FunctionTok{\#\#\#\# Subsubsection}
\end{Highlighting}
\end{Shaded}

\textbf{Add custom IDs for cross-referencing:}

\begin{Shaded}
\begin{Highlighting}[]
\FunctionTok{\#\# My Section \{\#sec{-}custom{-}id\}}
\end{Highlighting}
\end{Shaded}

Then reference it: \texttt{@sec-custom-id} produces
Section~\ref{sec-sections}

\section{Text Formatting}\label{sec-text-format}

\begin{longtable}[]{@{}ll@{}}
\caption{Text formatting options}\label{tbl-text-format}\tabularnewline
\toprule\noalign{}
Syntax & Output \\
\midrule\noalign{}
\endfirsthead
\toprule\noalign{}
Syntax & Output \\
\midrule\noalign{}
\endhead
\bottomrule\noalign{}
\endlastfoot
\texttt{**bold\ text**} & \textbf{bold text} \\
\texttt{*italic\ text*} & \emph{italic text} \\
\texttt{\textasciigrave{}code\textasciigrave{}} & \texttt{code} \\
\texttt{\textasciitilde{}\textasciitilde{}strikethrough\textasciitilde{}\textasciitilde{}}
& \st{strikethrough} \\
\texttt{H\textasciitilde{}2\textasciitilde{}O} (subscript) &
H\textsubscript{2}O \\
\texttt{X\^{}2\^{}} (superscript) & X\textsuperscript{2} \\
\end{longtable}

\section{Lists}\label{sec-lists}

\textbf{Numbered list:}

\begin{Shaded}
\begin{Highlighting}[]
\SpecialStringTok{1. }\NormalTok{First item}
\SpecialStringTok{2. }\NormalTok{Second item  }
\SpecialStringTok{3. }\NormalTok{Third item}
\end{Highlighting}
\end{Shaded}

Output:

\begin{enumerate}
\def\labelenumi{\arabic{enumi}.}
\tightlist
\item
  First item
\item
  Second item
\item
  Third item
\end{enumerate}

\textbf{Bullet list:}

\begin{Shaded}
\begin{Highlighting}[]
\SpecialStringTok{{-} }\NormalTok{Item one}
\SpecialStringTok{{-} }\NormalTok{Item two}
\SpecialStringTok{  {-} }\NormalTok{Nested item}
\end{Highlighting}
\end{Shaded}

Output:

\begin{itemize}
\tightlist
\item
  Item one
\item
  Item two

  \begin{itemize}
  \tightlist
  \item
    Nested item
  \end{itemize}
\end{itemize}

\section{Figures}\label{sec-figures}

\subsection{Images from Files}\label{images-from-files}

Place images in \texttt{figures/} directory:

\begin{Shaded}
\begin{Highlighting}[]
\AlertTok{![Caption text](figures/image.png)}\NormalTok{\{\#fig{-}label width=80\%\}}
\end{Highlighting}
\end{Shaded}

\textbf{Attributes:} - \texttt{\#fig-label} - ID for cross-referencing\\
- \texttt{width=80\%} - Width (percentage or absolute like \texttt{5in})
- \texttt{fig-align="center"} - Alignment (left, center, right)

\subsection{Diagrams with Mermaid}\label{diagrams-with-mermaid}

Mermaid diagrams allow you to create flowcharts, sequence diagrams, and
more using text-based syntax.

\textbf{Basic syntax example:}

\begin{Shaded}
\begin{Highlighting}[]
\NormalTok{::: \{\#fig{-}label\}}
\InformationTok{\textasciigrave{}\textasciigrave{}\textasciigrave{}\{mermaid\}}
\NormalTok{flowchart TD}
\NormalTok{    A[Start] {-}{-}\textgreater{} B\{Decision?\}}
\NormalTok{    B {-}{-}\textgreater{}|Yes| C[Process A]}
\NormalTok{    B {-}{-}\textgreater{}|No| D[Process B]}
\InformationTok{\textasciigrave{}\textasciigrave{}\textasciigrave{}}
\NormalTok{Caption for the diagram.}
\NormalTok{:::}
\end{Highlighting}
\end{Shaded}

\textbf{See Chapter~\ref{sec-methodology} for a working example.}

\subsection{Multiple Subfigures}\label{multiple-subfigures}

Create side-by-side figures:

\begin{Shaded}
\begin{Highlighting}[]
\NormalTok{::: \{\#fig{-}comparison layout{-}ncol=2\}}

\AlertTok{![Method A](figures/method{-}a.png)}\NormalTok{\{\#fig{-}method{-}a\}}

\AlertTok{![Method B](figures/method{-}b.png)}\NormalTok{\{\#fig{-}method{-}b\}}

\NormalTok{Comparison of two methods.}
\NormalTok{:::}
\end{Highlighting}
\end{Shaded}

Reference subfigures: \texttt{@fig-method-a} and \texttt{@fig-method-b}

\section{Tables}\label{sec-tables}

\subsection{Basic Table}\label{basic-table}

\textbf{Syntax:}

\begin{Shaded}
\begin{Highlighting}[]
\PreprocessorTok{|}\NormalTok{ Column 1 }\PreprocessorTok{|}\NormalTok{ Column 2 }\PreprocessorTok{|}\NormalTok{ Column 3 }\PreprocessorTok{|}
\PreprocessorTok{|{-}{-}{-}{-}{-}{-}{-}{-}{-}{-}|{-}{-}{-}{-}{-}{-}{-}{-}{-}{-}|{-}{-}{-}{-}{-}{-}{-}{-}{-}{-}|}
\PreprocessorTok{|}\NormalTok{ Data A   }\PreprocessorTok{|}\NormalTok{ Data B   }\PreprocessorTok{|}\NormalTok{ Data C   }\PreprocessorTok{|}
\PreprocessorTok{|}\NormalTok{ Data D   }\PreprocessorTok{|}\NormalTok{ Data E   }\PreprocessorTok{|}\NormalTok{ Data F   }\PreprocessorTok{|}

\NormalTok{: Table caption \{\#tbl{-}label\}}
\end{Highlighting}
\end{Shaded}

\textbf{Output:}

\begin{longtable}[]{@{}lll@{}}
\caption{Performance comparison of different
methods}\label{tbl-performance}\tabularnewline
\toprule\noalign{}
Method & Accuracy & F1-Score \\
\midrule\noalign{}
\endfirsthead
\toprule\noalign{}
Method & Accuracy & F1-Score \\
\midrule\noalign{}
\endhead
\bottomrule\noalign{}
\endlastfoot
Baseline & 0.82 & 0.79 \\
Proposed & 0.91 & 0.88 \\
State-of-art & 0.87 & 0.84 \\
\end{longtable}

\textbf{Reference:} \texttt{@tbl-performance} produces
Table~\ref{tbl-performance}

\subsection{Column Alignment}\label{column-alignment}

\begin{Shaded}
\begin{Highlighting}[]
\PreprocessorTok{|}\NormalTok{ Left }\PreprocessorTok{|}\NormalTok{ Center }\PreprocessorTok{|}\NormalTok{ Right }\PreprocessorTok{|}
\PreprocessorTok{|:{-}{-}{-}{-}{-}|:{-}{-}{-}{-}{-}{-}:|{-}{-}{-}{-}{-}{-}:|}
\PreprocessorTok{|}\NormalTok{ A    }\PreprocessorTok{|}\NormalTok{ B      }\PreprocessorTok{|}\NormalTok{ C     }\PreprocessorTok{|}
\end{Highlighting}
\end{Shaded}

Output:

\begin{longtable}[]{@{}lcr@{}}
\caption{Algorithm complexity}\label{tbl-complexity}\tabularnewline
\toprule\noalign{}
Algorithm & Complexity & Memory \\
\midrule\noalign{}
\endfirsthead
\toprule\noalign{}
Algorithm & Complexity & Memory \\
\midrule\noalign{}
\endhead
\bottomrule\noalign{}
\endlastfoot
Method A & O(n) & 10 MB \\
Method B & O(n log n) & 25 MB \\
\end{longtable}

\section{Equations}\label{sec-equations}

\subsection{Inline Equations}\label{inline-equations}

\textbf{Syntax:} \texttt{\$E\ =\ mc\^{}2\$}\\
\textbf{Output:} The famous equation \(E = mc^2\) shows energy-mass
equivalence.

\subsection{Display Equations}\label{display-equations}

\textbf{Single equation:}

\begin{Shaded}
\begin{Highlighting}[]
\NormalTok{$$}
\NormalTok{f(x) = \textbackslash{}int\_\{{-}\textbackslash{}infty\}\^{}\{\textbackslash{}infty\} e\^{}\{{-}x\^{}2\} dx}
\NormalTok{$$ \{\#eq{-}gaussian\}}
\end{Highlighting}
\end{Shaded}

\textbf{Output:}

\begin{equation}\phantomsection\label{eq-gaussian}{
f(x) = \int_{-\infty}^{\infty} e^{-x^2} dx
}\end{equation}

\textbf{Reference:} \texttt{@eq-gaussian} produces
Equation~\ref{eq-gaussian}

\subsection{Multi-line Equations}\label{multi-line-equations}

\textbf{Syntax:}

\begin{Shaded}
\begin{Highlighting}[]
\NormalTok{$$}
\NormalTok{\textbackslash{}begin\{aligned\}}
\NormalTok{\textbackslash{}nabla \textbackslash{}cdot \textbackslash{}mathbf\{E\} \&= \textbackslash{}frac\{\textbackslash{}rho\}\{\textbackslash{}epsilon\_0\} }\SpecialCharTok{\textbackslash{}\textbackslash{}}
\NormalTok{\textbackslash{}nabla \textbackslash{}cdot \textbackslash{}mathbf\{B\} \&= 0 }\SpecialCharTok{\textbackslash{}\textbackslash{}}
\NormalTok{\textbackslash{}nabla \textbackslash{}times \textbackslash{}mathbf\{E\} \&= {-}\textbackslash{}frac\{\textbackslash{}partial \textbackslash{}mathbf\{B\}\}\{\textbackslash{}partial t\} }\SpecialCharTok{\textbackslash{}\textbackslash{}}
\NormalTok{\textbackslash{}nabla \textbackslash{}times \textbackslash{}mathbf\{B\} \&= \textbackslash{}mu\_0\textbackslash{}mathbf\{J\} + \textbackslash{}mu\_0\textbackslash{}epsilon\_0\textbackslash{}frac\{\textbackslash{}partial \textbackslash{}mathbf\{E\}\}\{\textbackslash{}partial t\}}
\NormalTok{\textbackslash{}end\{aligned\}}
\NormalTok{$$ \{\#eq{-}maxwell\}}
\end{Highlighting}
\end{Shaded}

\textbf{Output:}

\begin{equation}\phantomsection\label{eq-maxwell}{
\begin{aligned}
\nabla \cdot \mathbf{E} &= \frac{\rho}{\epsilon_0} \\
\nabla \cdot \mathbf{B} &= 0 \\
\nabla \times \mathbf{E} &= -\frac{\partial \mathbf{B}}{\partial t} \\
\nabla \times \mathbf{B} &= \mu_0\mathbf{J} + \mu_0\epsilon_0\frac{\partial \mathbf{E}}{\partial t}
\end{aligned}
}\end{equation}

\subsection{Matrices}\label{matrices}

\begin{Shaded}
\begin{Highlighting}[]
\NormalTok{$$}
\NormalTok{\textbackslash{}mathbf\{A\} = \textbackslash{}begin\{bmatrix\}}
\NormalTok{a\_\{11\} \& a\_\{12\} \& a\_\{13\} }\SpecialCharTok{\textbackslash{}\textbackslash{}}
\NormalTok{a\_\{21\} \& a\_\{22\} \& a\_\{23\} }\SpecialCharTok{\textbackslash{}\textbackslash{}}
\NormalTok{a\_\{31\} \& a\_\{32\} \& a\_\{33\}}
\NormalTok{\textbackslash{}end\{bmatrix\}}
\NormalTok{$$ \{\#eq{-}matrix\}}
\end{Highlighting}
\end{Shaded}

Output:

\begin{equation}\phantomsection\label{eq-matrix}{
\mathbf{A} = \begin{bmatrix}
a_{11} & a_{12} & a_{13} \\
a_{21} & a_{22} & a_{23} \\
a_{31} & a_{32} & a_{33}
\end{bmatrix}
}\end{equation}

\section{Code Blocks}\label{sec-code}

\subsection{Syntax Highlighted Code}\label{syntax-highlighted-code}

\textbf{Python example:}

\begin{Shaded}
\begin{Highlighting}[]
\ImportTok{import}\NormalTok{ numpy }\ImportTok{as}\NormalTok{ np}
\ImportTok{import}\NormalTok{ matplotlib.pyplot }\ImportTok{as}\NormalTok{ plt}

\CommentTok{\# Generate data}
\NormalTok{x }\OperatorTok{=}\NormalTok{ np.linspace(}\DecValTok{0}\NormalTok{, }\DecValTok{10}\NormalTok{, }\DecValTok{100}\NormalTok{)}
\NormalTok{y }\OperatorTok{=}\NormalTok{ np.sin(x)}

\CommentTok{\# Plot}
\NormalTok{plt.plot(x, y)}
\NormalTok{plt.xlabel(}\StringTok{\textquotesingle{}x\textquotesingle{}}\NormalTok{)}
\NormalTok{plt.ylabel(}\StringTok{\textquotesingle{}sin(x)\textquotesingle{}}\NormalTok{)}
\NormalTok{plt.show()}
\end{Highlighting}
\end{Shaded}

\textbf{Other languages:}

\begin{Shaded}
\begin{Highlighting}[]
\InformationTok{\textasciigrave{}\textasciigrave{}\textasciigrave{}r}
\CommentTok{\# R code}
\NormalTok{data }\OtherTok{\textless{}{-}} \FunctionTok{c}\NormalTok{(}\DecValTok{1}\NormalTok{, }\DecValTok{2}\NormalTok{, }\DecValTok{3}\NormalTok{, }\DecValTok{4}\NormalTok{, }\DecValTok{5}\NormalTok{)}
\FunctionTok{mean}\NormalTok{(data)}
\InformationTok{\textasciigrave{}\textasciigrave{}\textasciigrave{}}

\InformationTok{\textasciigrave{}\textasciigrave{}\textasciigrave{}javascript}
\CommentTok{// JavaScript}
\NormalTok{function }\FunctionTok{factorial}\OperatorTok{(}\NormalTok{n}\OperatorTok{)} \OperatorTok{\{}
  \ControlFlowTok{return}\NormalTok{ n }\OperatorTok{\textless{}=} \DecValTok{1} \OperatorTok{?} \DecValTok{1} \OperatorTok{:}\NormalTok{ n }\OperatorTok{*} \FunctionTok{factorial}\OperatorTok{(}\NormalTok{n }\OperatorTok{{-}} \DecValTok{1}\OperatorTok{);}
\OperatorTok{\}}
\NormalTok{\textasciigrave{}\textasciigrave{}\textasciigrave{}}
\end{Highlighting}
\end{Shaded}

\section{Cross-References}\label{sec-cross-ref}

\begin{longtable}[]{@{}lll@{}}
\caption{Cross-reference syntax}\label{tbl-crossref}\tabularnewline
\toprule\noalign{}
Type & Syntax & Example Output \\
\midrule\noalign{}
\endfirsthead
\toprule\noalign{}
Type & Syntax & Example Output \\
\midrule\noalign{}
\endhead
\bottomrule\noalign{}
\endlastfoot
Section & \texttt{@sec-label} & Section~\ref{sec-figures} \\
Figure & \texttt{@fig-label} & \textbf{?@fig-example} \\
Table & \texttt{@tbl-label} & Table~\ref{tbl-performance} \\
Equation & \texttt{@eq-label} & Equation~\ref{eq-gaussian} \\
\end{longtable}

\textbf{Multiple references:}

\begin{Shaded}
\begin{Highlighting}[]
\NormalTok{See @sec{-}figures and @sec{-}tables for examples.}
\NormalTok{Figures @fig{-}example and @tbl{-}example show basic elements.}
\end{Highlighting}
\end{Shaded}

\section{Citations}\label{sec-citations}

Add references to \texttt{references.bib} or \texttt{MyLibrary.bib}:

\begin{Shaded}
\begin{Highlighting}[]
\VariableTok{@article}\NormalTok{\{}\OtherTok{smith2023}\NormalTok{,}
  \DataTypeTok{author}\NormalTok{ = \{Smith, John\},}
  \DataTypeTok{title}\NormalTok{ = \{Example Paper\},}
  \DataTypeTok{journal}\NormalTok{ = \{Journal Name\},}
  \DataTypeTok{year}\NormalTok{ = \{2023\},}
  \DataTypeTok{volume}\NormalTok{ = \{10\},}
  \DataTypeTok{pages}\NormalTok{ = \{1{-}{-}10\}}
\NormalTok{\}}
\end{Highlighting}
\end{Shaded}

\textbf{Citation syntax:}

\begin{longtable}[]{@{}ll@{}}
\caption{Citation examples}\label{tbl-citations}\tabularnewline
\toprule\noalign{}
Syntax & Output \\
\midrule\noalign{}
\endfirsthead
\toprule\noalign{}
Syntax & Output \\
\midrule\noalign{}
\endhead
\bottomrule\noalign{}
\endlastfoot
\texttt{@einstein1905} & Einstein (1905) \\
\texttt{{[}@einstein1905{]}} & (Einstein, 1905) \\
\texttt{{[}@einstein1905;\ @knuth1984{]}} & (Einstein, 1905; Knuth,
1984) \\
\texttt{{[}@einstein1905,\ p.\ 23{]}} & (Einstein, 1905, p.~23) \\
\end{longtable}

\textbf{Example in text:}

The theory of relativity \textcite{einstein1905} revolutionized physics.
Many researchers have studied this \autocite{knuth1984,lamport1994}.

\section{Algorithms}\label{sec-intro-algorithms}

For algorithms, use the pseudocode environment:

\begin{algorithm}[H]
\caption{Binary Search Algorithm}
\label{alg-binary-search}
\begin{algorithmic}[1]
\Require Sorted array $A$, target value $x$
\Ensure Index of $x$ in $A$, or $-1$ if not found

\State $low \leftarrow 0$
\State $high \leftarrow \text{length}(A) - 1$
\While{$low \leq high$}
    \State $mid \leftarrow \lfloor(low + high) / 2\rfloor$
    \If{$A[mid] = x$}
        \State \Return $mid$
    \ElsIf{$A[mid] < x$}
        \State $low \leftarrow mid + 1$
    \Else
        \State $high \leftarrow mid - 1$
    \EndIf
\EndWhile
\State \Return $-1$
\end{algorithmic}
\end{algorithm}

\section{Important Notes for
Synopsis}\label{important-notes-for-synopsis}

\subsection{Required Elements}\label{required-elements}

Your synopsis should include:

\begin{enumerate}
\def\labelenumi{\arabic{enumi}.}
\tightlist
\item
  \textbf{Title and metadata} (in \texttt{\_quarto.yml})
\item
  \textbf{Abstract} (add \texttt{\#\#\ Abstract\ \{.unnumbered\}}
  section)
\item
  \textbf{Introduction} with objectives
\item
  \textbf{Literature review} with citations
\item
  \textbf{Methodology} with figures and equations
\item
  \textbf{Expected results} (or preliminary results)
\item
  \textbf{References} (automatically generated from citations)
\end{enumerate}

\subsection{Best Practices}\label{best-practices}

\begin{itemize}
\tightlist
\item
  Use consistent section numbering with \texttt{\{\#sec-label\}} IDs
\item
  Number all figures, tables, and equations
\item
  Reference them in text using \texttt{@} notation
\item
  Add captions to all figures and tables
\item
  Cite relevant literature throughout
\item
  Use diagrams to visualize concepts
\end{itemize}

\subsection{Building the Document}\label{building-the-document}

\textbf{Render to PDF:}

\begin{Shaded}
\begin{Highlighting}[]
\ExtensionTok{quarto}\NormalTok{ render }\AttributeTok{{-}{-}to}\NormalTok{ manit{-}pre{-}thesis{-}synopsis{-}pdf}
\end{Highlighting}
\end{Shaded}

\textbf{Preview while editing:}

\begin{Shaded}
\begin{Highlighting}[]
\ExtensionTok{quarto}\NormalTok{ preview}
\end{Highlighting}
\end{Shaded}

Now write your actual synopsis content, replacing these tutorial
sections with your research content!

\bookmarksetup{startatroot}

\chapter{Literature Review}\label{sec-literature-review}

This chapter demonstrates citation techniques and literature
organization.

\section{How to Add References}\label{sec-add-refs}

\subsection{Step 1: Add to BibTeX File}\label{step-1-add-to-bibtex-file}

Open \texttt{references.bib} or \texttt{MyLibrary.bib} and add entries:

\begin{Shaded}
\begin{Highlighting}[]
\VariableTok{@article}\NormalTok{\{}\OtherTok{author2023}\NormalTok{,}
  \DataTypeTok{author}\NormalTok{ = \{Last, First and Second, Author\},}
  \DataTypeTok{title}\NormalTok{ = \{Article Title\},}
  \DataTypeTok{journal}\NormalTok{ = \{Journal Name\},}
  \DataTypeTok{year}\NormalTok{ = \{2023\},}
  \DataTypeTok{volume}\NormalTok{ = \{10\},}
  \DataTypeTok{number}\NormalTok{ = \{2\},}
  \DataTypeTok{pages}\NormalTok{ = \{123{-}{-}145\},}
  \DataTypeTok{doi}\NormalTok{ = \{10.1234/journal.2023.001\}}
\NormalTok{\}}

\VariableTok{@inproceedings}\NormalTok{\{}\OtherTok{author2022}\NormalTok{,}
  \DataTypeTok{author}\NormalTok{ = \{Author, Name\},}
  \DataTypeTok{title}\NormalTok{ = \{Conference Paper Title\},}
  \DataTypeTok{booktitle}\NormalTok{ = \{Proceedings of Conference Name\},}
  \DataTypeTok{year}\NormalTok{ = \{2022\},}
  \DataTypeTok{pages}\NormalTok{ = \{45{-}{-}58\},}
  \DataTypeTok{publisher}\NormalTok{ = \{Publisher\}}
\NormalTok{\}}

\VariableTok{@book}\NormalTok{\{}\OtherTok{author2021}\NormalTok{,}
  \DataTypeTok{author}\NormalTok{ = \{Author, Name\},}
  \DataTypeTok{title}\NormalTok{ = \{Book Title\},}
  \DataTypeTok{publisher}\NormalTok{ = \{Publisher Name\},}
  \DataTypeTok{year}\NormalTok{ = \{2021\},}
  \DataTypeTok{edition}\NormalTok{ = \{2nd\}}
\NormalTok{\}}
\end{Highlighting}
\end{Shaded}

\subsection{Step 2: Get BibTeX Entries}\label{step-2-get-bibtex-entries}

\textbf{From Google Scholar:} 1. Search for paper 2. Click ``Cite''\\
3. Select ``BibTeX'' 4. Copy and paste into your \texttt{.bib} file

\textbf{From DOI:} - Visit https://doi2bib.org - Enter DOI - Get BibTeX
format

\textbf{From Reference Managers:} - Zotero, Mendeley, EndNote all export
BibTeX

\section{Citation Syntax}\label{sec-cite-syntax}

\subsection{Basic Citations}\label{basic-citations}

\begin{longtable}[]{@{}
  >{\raggedright\arraybackslash}p{(\linewidth - 4\tabcolsep) * \real{0.2727}}
  >{\raggedright\arraybackslash}p{(\linewidth - 4\tabcolsep) * \real{0.3636}}
  >{\raggedright\arraybackslash}p{(\linewidth - 4\tabcolsep) * \real{0.3636}}@{}}
\caption{Citation syntax and
output}\label{tbl-cite-syntax}\tabularnewline
\toprule\noalign{}
\begin{minipage}[b]{\linewidth}\raggedright
Type
\end{minipage} & \begin{minipage}[b]{\linewidth}\raggedright
Syntax
\end{minipage} & \begin{minipage}[b]{\linewidth}\raggedright
Output
\end{minipage} \\
\midrule\noalign{}
\endfirsthead
\toprule\noalign{}
\begin{minipage}[b]{\linewidth}\raggedright
Type
\end{minipage} & \begin{minipage}[b]{\linewidth}\raggedright
Syntax
\end{minipage} & \begin{minipage}[b]{\linewidth}\raggedright
Output
\end{minipage} \\
\midrule\noalign{}
\endhead
\bottomrule\noalign{}
\endlastfoot
Narrative & \texttt{@smith2023machine} & Smith and Doe (2023) \\
Parenthetical & \texttt{{[}@smith2023machine{]}} & (Smith and Doe,
2023) \\
Multiple & \texttt{{[}@smith2023machine;\ @jones2022deep{]}} & (Smith
and Doe, 2023; Jones and Brown, 2022) \\
With page & \texttt{{[}@smith2023machine,\ p.\ 42{]}} & (Smith and Doe,
2023, p.~42) \\
Suppress author & \texttt{{[}-@smith2023machine{]}} & (2023) \\
\end{longtable}

\subsection{Examples in Context}\label{examples-in-context}

\textbf{Narrative citation:}

Recent work by \textcite{smith2023machine} shows that machine learning
approaches improve accuracy by 15\%.

\textbf{Parenthetical citation:}

Machine learning has shown significant improvements in recent years
\autocite{jones2022deep}.

\textbf{Multiple citations:}

Several studies have investigated this phenomenon
\autocite{smith2023machine,jones2022deep,wilson2023framework}.

\textbf{With page numbers:}

As noted by \textcite[95]{kumar2022analysis}, statistical methods are
crucial.

\section{Literature Organization}\label{sec-lit-organization}

\subsection{Organizing by Topic}\label{organizing-by-topic}

Organize your literature review by themes, not chronologically:

\subsubsection{Machine Learning
Approaches}\label{machine-learning-approaches}

Early work in machine learning \textcite{russell2020artificial}
established foundational algorithms. Recent advances by
\textcite{smith2023machine} and \textcite{jones2022deep} have improved
performance significantly. The framework proposed by
\textcite{wilson2023framework} provides a unified approach.

\subsubsection{Statistical Methods}\label{statistical-methods}

Traditional statistical approaches \textcite{kumar2022analysis} remain
relevant. However, modern techniques \textcite{martin2023algorithms}
offer better scalability for large datasets.

\subsubsection{Distributed Computing}\label{distributed-computing}

The rise of big data necessitates distributed frameworks
\textcite{wang2022distributed}. These systems enable processing of
massive datasets efficiently.

\subsection{Identifying Research Gaps}\label{identifying-research-gaps}

After reviewing the literature, identify gaps:

\begin{enumerate}
\def\labelenumi{\arabic{enumi}.}
\item
  \textbf{Gap 1:} While \textcite{smith2023machine} achieved 85\%
  accuracy, their method requires extensive training data.
\item
  \textbf{Gap 2:} Existing approaches
  \autocite{jones2022deep,wilson2023framework} have not addressed
  real-time constraints.
\item
  \textbf{Gap 3:} As noted by \textcite{kumar2022analysis}, scalability
  remains an open challenge.
\end{enumerate}

\section{Comparison Tables}\label{sec-lit-comparison}

Summarize related work in tables:

\begin{longtable}[]{@{}lllll@{}}
\caption{Comparison of related
work}\label{tbl-lit-comparison}\tabularnewline
\toprule\noalign{}
Study & Method & Dataset & Accuracy & Year \\
\midrule\noalign{}
\endfirsthead
\toprule\noalign{}
Study & Method & Dataset & Accuracy & Year \\
\midrule\noalign{}
\endhead
\bottomrule\noalign{}
\endlastfoot
\textcite{smith2023machine} & Deep Learning & ImageNet & 91.2\% &
2023 \\
\textcite{jones2022deep} & Neural Network & CIFAR-10 & 89.5\% & 2022 \\
\textcite{patel2023neural} & CNN & Custom & 87.8\% & 2023 \\
\textcite{martin2023algorithms} & Random Forest & UCI & 83.2\% & 2023 \\
\end{longtable}

Reference the table: See Table~\ref{tbl-lit-comparison} for a detailed
comparison.

\section{Critical Analysis}\label{sec-critical-analysis}

Don't just summarize - analyze critically:

\textbf{Strengths:} - The method by \textcite{smith2023machine} achieves
high accuracy - Approach of \textcite{wilson2023framework} is
computationally efficient - \textcite{jones2022deep} provides strong
theoretical foundations

\textbf{Limitations:} - Most studies
\autocite{smith2023machine,jones2022deep} use limited datasets -
Real-time performance not addressed by \textcite{wilson2023framework}\\
- Scalability concerns raised by \textcite{kumar2022analysis} remain
unresolved

\textbf{Research Opportunities:} Based on gaps identified above, this
work proposes to\ldots{}

\section{Connection to Your Work}\label{sec-connection}

End with how your work fits in:

While existing approaches
\autocite{smith2023machine,jones2022deep,wilson2023framework} have made
significant progress, they share common limitations. This research
addresses these gaps by proposing a novel framework that combines the
strengths of \textcite{smith2023machine} and
\textcite{wilson2023framework} while overcoming the scalability issues
identified by \textcite{kumar2022analysis}.

The methodology described in Chapter~\ref{sec-methodology} builds upon
these foundations to develop an improved approach.

\textbf{Remember:} - Cite sources for all claims - Group by themes, not
chronologically - Critically analyze, don't just summarize\\
- Identify specific gaps your work addresses - Use tables and diagrams
to clarify comparisons - Connect literature review to your methodology

\bookmarksetup{startatroot}

\chapter{Methodology}\label{sec-methodology}

This chapter demonstrates how to present your research methodology with
proper figures, tables, equations, and algorithms.

\section{Figures and Diagrams}\label{sec-figures-diagrams}

\subsection{Flowcharts with Mermaid}\label{flowcharts-with-mermaid}

For process flows and workflows, use Mermaid diagrams:

\begin{figure}

\centering{

\includegraphics[width=2.44in,height=8.37in]{Chapters/03_methodology_files/figure-latex/mermaid-figure-1.png}

}

\caption{\label{fig-workflow}Methodology workflow.}

\end{figure}%

\textbf{Reference:} See Figure~\ref{fig-workflow} for the overall
process.

\section{Mathematical Formulation}\label{sec-math-formulation}

\subsection{Problem Definition}\label{problem-definition}

Let \(\mathbf{X} = \{x_1, x_2, \ldots, x_n\}\) be the input dataset
where \(x_i \in \mathbb{R}^d\). The objective is to learn a function:

\begin{equation}\phantomsection\label{eq-objective}{
f: \mathbb{R}^d \rightarrow \mathbb{C}
}\end{equation}

where \(\mathbb{C} = \{c_1, c_2, \ldots, c_k\}\) is the set of \(k\)
classes.

\subsection{Preprocessing}\label{preprocessing}

\textbf{Normalization:} Each feature is normalized using z-score
normalization:

\begin{equation}\phantomsection\label{eq-normalization}{
x'_i = \frac{x_i - \mu}{\sigma}
}\end{equation}

where \(\mu\) is the mean and \(\sigma\) is the standard deviation.

\textbf{Feature extraction:} Principal Component Analysis (PCA)
transforms data:

\begin{equation}\phantomsection\label{eq-pca}{
\mathbf{Y} = \mathbf{X} \mathbf{W}
}\end{equation}

where \(\mathbf{W}\) is the matrix of eigenvectors.

\subsection{Model Formulation}\label{model-formulation}

The classification model is defined as:

\begin{equation}\phantomsection\label{eq-classification}{
\hat{y} = \arg\max_{c \in \mathbb{C}} P(y = c | \mathbf{x}, \boldsymbol{\theta})
}\end{equation}

where \(\boldsymbol{\theta}\) represents model parameters.

\textbf{Loss function:}

\begin{equation}\phantomsection\label{eq-loss}{
\mathcal{L}(\boldsymbol{\theta}) = -\frac{1}{N} \sum_{i=1}^{N} \sum_{c=1}^{k} y_{ic} \log(p_{ic})
}\end{equation}

where \(y_{ic}\) is the ground truth and \(p_{ic}\) is the predicted
probability.

\subsection{Optimization}\label{optimization}

Parameters are optimized using gradient descent:

\begin{equation}\phantomsection\label{eq-gradient-descent}{
\boldsymbol{\theta}_{t+1} = \boldsymbol{\theta}_t - \alpha \nabla_{\boldsymbol{\theta}} \mathcal{L}(\boldsymbol{\theta}_t)
}\end{equation}

where \(\alpha\) is the learning rate.

\textbf{Convergence criterion:}

\begin{equation}\phantomsection\label{eq-convergence}{
\|\nabla_{\boldsymbol{\theta}} \mathcal{L}(\boldsymbol{\theta})\| < \epsilon
}\end{equation}

where \(\epsilon = 10^{-6}\) is the tolerance threshold.

\section{Algorithms}\label{sec-algorithms}

Algorithms can be written using the pseudocode environment for
professional formatting.

\subsection{Training Algorithm}\label{training-algorithm}

\begin{algorithm}[H]
\caption{Model Training Algorithm}
\label{alg-training}
\begin{algorithmic}[1]
\Require Training data $X$, labels $y$, learning rate $\alpha$, epochs $E$
\Ensure Trained model parameters $\theta$

\State Initialize $\theta$ randomly
\For{$epoch = 1$ to $E$}
    \State Shuffle training data
    \ForAll{batch $B$ in $X$}
        \State Compute predictions: $\hat{y} = f(B; \theta)$
        \State Compute loss: $L = \text{Loss}(\hat{y}, y)$
        \State Compute gradients: $g = \nabla_{\theta} L$
        \State Update parameters: $\theta \leftarrow \theta - \alpha \cdot g$
    \EndFor
    \State Validate on validation set
    \If{validation\_loss $<$ best\_loss}
        \State Save $\theta$ as best\_model
    \EndIf
\EndFor
\State \Return best\_model
\end{algorithmic}
\end{algorithm}

\subsection{Prediction Algorithm}\label{prediction-algorithm}

\begin{algorithm}[H]
\caption{Inference Algorithm}
\label{alg-inference}
\begin{algorithmic}[1]
\Require Test sample $x$, trained model $\theta$
\Ensure Predicted class $\hat{c}$ and confidence score

\State Preprocess $x$ using normalization (Eq. @eq-normalization)
\State Extract features: $x' = \text{FeatureExtract}(x)$
\State Compute class probabilities:
\ForAll{class $c$ in $C$}
    \State $p_c = P(y = c | x'; \theta)$
\EndFor
\State Find predicted class: $\hat{c} = \arg\max(p_c)$
\State Compute confidence: $\text{conf} = \max(p_c)$
\State \Return $\hat{c}$, conf
\end{algorithmic}
\end{algorithm}

\section{Experimental Setup}\label{sec-experimental-setup}

\subsection{Datasets}\label{datasets}

\begin{longtable}[]{@{}lllll@{}}
\caption{Datasets used in
experiments}\label{tbl-datasets}\tabularnewline
\toprule\noalign{}
Dataset & Samples & Features & Classes & Split \\
\midrule\noalign{}
\endfirsthead
\toprule\noalign{}
Dataset & Samples & Features & Classes & Split \\
\midrule\noalign{}
\endhead
\bottomrule\noalign{}
\endlastfoot
Dataset A & 10,000 & 128 & 10 & 70/15/15 \\
Dataset B & 25,000 & 256 & 5 & 80/10/10 \\
Dataset C & 50,000 & 512 & 20 & 75/12.5/12.5 \\
\end{longtable}

Split indicates train/validation/test percentages.

\subsection{Hyperparameters}\label{hyperparameters}

\begin{longtable}[]{@{}lll@{}}
\caption{Hyperparameter
configuration}\label{tbl-hyperparams}\tabularnewline
\toprule\noalign{}
Parameter & Value & Description \\
\midrule\noalign{}
\endfirsthead
\toprule\noalign{}
Parameter & Value & Description \\
\midrule\noalign{}
\endhead
\bottomrule\noalign{}
\endlastfoot
Learning rate (α) & 0.001 & Initial learning rate \\
Batch size & 32 & Training batch size \\
Epochs (E) & 100 & Maximum training epochs \\
Dropout & 0.3 & Dropout probability \\
Optimizer & Adam & Optimization algorithm \\
Weight decay & 1e-4 & L2 regularization \\
\end{longtable}

See Table~\ref{tbl-hyperparams} for complete parameter settings.

\subsection{Evaluation Metrics}\label{evaluation-metrics}

\textbf{Accuracy:}

\begin{equation}\phantomsection\label{eq-accuracy}{
\text{Accuracy} = \frac{\text{TP} + \text{TN}}{\text{TP} + \text{TN} + \text{FP} + \text{FN}}
}\end{equation}

\textbf{Precision:}

\begin{equation}\phantomsection\label{eq-precision}{
\text{Precision} = \frac{\text{TP}}{\text{TP} + \text{FP}}
}\end{equation}

\textbf{Recall:}

\begin{equation}\phantomsection\label{eq-recall}{
\text{Recall} = \frac{\text{TP}}{\text{TP} + \text{FN}}
}\end{equation}

\textbf{F1-Score:}

\begin{equation}\phantomsection\label{eq-f1}{
\text{F1} = 2 \cdot \frac{\text{Precision} \cdot \text{Recall}}{\text{Precision} + \text{Recall}}
}\end{equation}

\section{Implementation Details}\label{sec-implementation}

\subsection{Software and Libraries}\label{software-and-libraries}

\textbf{Programming environment:}

\begin{Shaded}
\begin{Highlighting}[]
\CommentTok{\# Python version}
\NormalTok{Python }\FloatTok{3.10.12}

\CommentTok{\# Key libraries}
\ImportTok{import}\NormalTok{ numpy }\ImportTok{as}\NormalTok{ np          }\CommentTok{\# v1.24.3}
\ImportTok{import}\NormalTok{ pandas }\ImportTok{as}\NormalTok{ pd         }\CommentTok{\# v2.0.2}
\ImportTok{import}\NormalTok{ scikit}\OperatorTok{{-}}\NormalTok{learn }\ImportTok{as}\NormalTok{ sklearn  }\CommentTok{\# v1.2.2}
\ImportTok{import}\NormalTok{ tensorflow }\ImportTok{as}\NormalTok{ tf     }\CommentTok{\# v2.13.0}
\end{Highlighting}
\end{Shaded}

\textbf{Hardware specifications:}

\begin{longtable}[]{@{}ll@{}}
\caption{Hardware configuration}\label{tbl-hardware}\tabularnewline
\toprule\noalign{}
Component & Specification \\
\midrule\noalign{}
\endfirsthead
\toprule\noalign{}
Component & Specification \\
\midrule\noalign{}
\endhead
\bottomrule\noalign{}
\endlastfoot
CPU & Intel Core i7-12700K \\
RAM & 32 GB DDR4 \\
GPU & NVIDIA RTX 3090 (24GB) \\
Storage & 1TB NVMe SSD \\
\end{longtable}

\subsection{Model Architecture}\label{model-architecture}

\textbf{Network structure:}

\begin{Shaded}
\begin{Highlighting}[]
\NormalTok{model }\OperatorTok{=}\NormalTok{ Sequential([}
\NormalTok{    Dense(}\DecValTok{256}\NormalTok{, activation}\OperatorTok{=}\StringTok{\textquotesingle{}relu\textquotesingle{}}\NormalTok{, input\_dim}\OperatorTok{=}\DecValTok{128}\NormalTok{),}
\NormalTok{    Dropout(}\FloatTok{0.3}\NormalTok{),}
\NormalTok{    Dense(}\DecValTok{128}\NormalTok{, activation}\OperatorTok{=}\StringTok{\textquotesingle{}relu\textquotesingle{}}\NormalTok{),}
\NormalTok{    Dropout(}\FloatTok{0.3}\NormalTok{),}
\NormalTok{    Dense(}\DecValTok{64}\NormalTok{, activation}\OperatorTok{=}\StringTok{\textquotesingle{}relu\textquotesingle{}}\NormalTok{),}
\NormalTok{    Dense(num\_classes, activation}\OperatorTok{=}\StringTok{\textquotesingle{}softmax\textquotesingle{}}\NormalTok{)}
\NormalTok{])}
\end{Highlighting}
\end{Shaded}

\textbf{Layer dimensions:}

\begin{equation}\phantomsection\label{eq-architecture}{
\begin{aligned}
\text{Input} &: \mathbb{R}^{128} \\
\text{Hidden}_1 &: \mathbb{R}^{256} \\
\text{Hidden}_2 &: \mathbb{R}^{128} \\
\text{Hidden}_3 &: \mathbb{R}^{64} \\
\text{Output} &: \mathbb{R}^{k}
\end{aligned}
}\end{equation}

\section{Summary}\label{summary}

This chapter presented:

\begin{enumerate}
\def\labelenumi{\arabic{enumi}.}
\tightlist
\item
  System architecture and diagrams
\item
  Mathematical formulation with equations
\item
  Training and inference algorithms
\item
  Experimental setup with datasets and hyperparameters
\end{enumerate}

Results and discussion are presented in Chapter~\ref{sec-results}.

\bookmarksetup{startatroot}

\chapter{Results and Discussion}\label{sec-results}

This chapter demonstrates various table types and result presentation
formats.

\section{Basic Tables}\label{sec-basic-tables}

\subsection{Simple Table}\label{simple-table}

\begin{longtable}[]{@{}ll@{}}
\caption{Experimental results}\label{tbl-results}\tabularnewline
\toprule\noalign{}
Metric & Value \\
\midrule\noalign{}
\endfirsthead
\toprule\noalign{}
Metric & Value \\
\midrule\noalign{}
\endhead
\bottomrule\noalign{}
\endlastfoot
Accuracy & 89.3\% \\
Precision & 87.5\% \\
Recall & 91.2\% \\
F1-Score & 88.3\% \\
\end{longtable}

Reference: Table~\ref{tbl-results} shows the performance metrics.

\section{Table Alignment}\label{sec-table-align}

Control column alignment:

\begin{Shaded}
\begin{Highlighting}[]
\PreprocessorTok{|}\NormalTok{ Left }\PreprocessorTok{|}\NormalTok{ Center }\PreprocessorTok{|}\NormalTok{ Right }\PreprocessorTok{|}
\PreprocessorTok{|:{-}{-}{-}{-}{-}|:{-}{-}{-}{-}{-}{-}:|{-}{-}{-}{-}{-}{-}:|}
\PreprocessorTok{|}\NormalTok{ A    }\PreprocessorTok{|}\NormalTok{ B      }\PreprocessorTok{|}\NormalTok{ C     }\PreprocessorTok{|}
\end{Highlighting}
\end{Shaded}

\begin{longtable}[]{@{}lcr@{}}
\caption{Performance comparison with
alignment}\label{tbl-aligned}\tabularnewline
\toprule\noalign{}
Algorithm & Time (ms) & Accuracy \\
\midrule\noalign{}
\endfirsthead
\toprule\noalign{}
Algorithm & Time (ms) & Accuracy \\
\midrule\noalign{}
\endhead
\bottomrule\noalign{}
\endlastfoot
Method A & 120 & 85.3\% \\
Method B & 150 & 89.7\% \\
Method C & 180 & 92.1\% \\
\end{longtable}

\section{Multi-line Tables}\label{sec-multiline}

For complex content:

\begin{longtable}[]{@{}
  >{\raggedright\arraybackslash}p{(\linewidth - 6\tabcolsep) * \real{0.2424}}
  >{\raggedright\arraybackslash}p{(\linewidth - 6\tabcolsep) * \real{0.3939}}
  >{\raggedright\arraybackslash}p{(\linewidth - 6\tabcolsep) * \real{0.1818}}
  >{\raggedright\arraybackslash}p{(\linewidth - 6\tabcolsep) * \real{0.1818}}@{}}
\caption{Detailed comparison of
methods}\label{tbl-multiline}\tabularnewline
\toprule\noalign{}
\begin{minipage}[b]{\linewidth}\raggedright
Method
\end{minipage} & \begin{minipage}[b]{\linewidth}\raggedright
Description
\end{minipage} & \begin{minipage}[b]{\linewidth}\raggedright
Pros
\end{minipage} & \begin{minipage}[b]{\linewidth}\raggedright
Cons
\end{minipage} \\
\midrule\noalign{}
\endfirsthead
\toprule\noalign{}
\begin{minipage}[b]{\linewidth}\raggedright
Method
\end{minipage} & \begin{minipage}[b]{\linewidth}\raggedright
Description
\end{minipage} & \begin{minipage}[b]{\linewidth}\raggedright
Pros
\end{minipage} & \begin{minipage}[b]{\linewidth}\raggedright
Cons
\end{minipage} \\
\midrule\noalign{}
\endhead
\bottomrule\noalign{}
\endlastfoot
Deep Learning & Uses neural networks with multiple layers & High
accuracy, learns features automatically & Requires large dataset,
computationally expensive \\
Random Forest & Ensemble of decision trees & Handles non-linear data,
robust & Can overfit, slow for large datasets \\
SVM & Finds optimal hyperplane & Effective in high dimensions &
Sensitive to parameters \\
\end{longtable}

\section{Wide Tables}\label{sec-wide-tables}

For tables with many columns:

\begin{longtable}[]{@{}
  >{\raggedright\arraybackslash}p{(\linewidth - 18\tabcolsep) * \real{0.0938}}
  >{\raggedright\arraybackslash}p{(\linewidth - 18\tabcolsep) * \real{0.1146}}
  >{\raggedright\arraybackslash}p{(\linewidth - 18\tabcolsep) * \real{0.1146}}
  >{\raggedright\arraybackslash}p{(\linewidth - 18\tabcolsep) * \real{0.1042}}
  >{\raggedright\arraybackslash}p{(\linewidth - 18\tabcolsep) * \real{0.0938}}
  >{\raggedright\arraybackslash}p{(\linewidth - 18\tabcolsep) * \real{0.1042}}
  >{\raggedright\arraybackslash}p{(\linewidth - 18\tabcolsep) * \real{0.1042}}
  >{\raggedright\arraybackslash}p{(\linewidth - 18\tabcolsep) * \real{0.1042}}
  >{\raggedright\arraybackslash}p{(\linewidth - 18\tabcolsep) * \real{0.1042}}
  >{\raggedright\arraybackslash}p{(\linewidth - 18\tabcolsep) * \real{0.0625}}@{}}
\caption{Comprehensive experimental
results}\label{tbl-wide}\tabularnewline
\toprule\noalign{}
\begin{minipage}[b]{\linewidth}\raggedright
Dataset
\end{minipage} & \begin{minipage}[b]{\linewidth}\raggedright
Train Size
\end{minipage} & \begin{minipage}[b]{\linewidth}\raggedright
Test Size
\end{minipage} & \begin{minipage}[b]{\linewidth}\raggedright
Features
\end{minipage} & \begin{minipage}[b]{\linewidth}\raggedright
Classes
\end{minipage} & \begin{minipage}[b]{\linewidth}\raggedright
Baseline
\end{minipage} & \begin{minipage}[b]{\linewidth}\raggedright
Method 1
\end{minipage} & \begin{minipage}[b]{\linewidth}\raggedright
Method 2
\end{minipage} & \begin{minipage}[b]{\linewidth}\raggedright
Method 3
\end{minipage} & \begin{minipage}[b]{\linewidth}\raggedright
Best
\end{minipage} \\
\midrule\noalign{}
\endfirsthead
\toprule\noalign{}
\begin{minipage}[b]{\linewidth}\raggedright
Dataset
\end{minipage} & \begin{minipage}[b]{\linewidth}\raggedright
Train Size
\end{minipage} & \begin{minipage}[b]{\linewidth}\raggedright
Test Size
\end{minipage} & \begin{minipage}[b]{\linewidth}\raggedright
Features
\end{minipage} & \begin{minipage}[b]{\linewidth}\raggedright
Classes
\end{minipage} & \begin{minipage}[b]{\linewidth}\raggedright
Baseline
\end{minipage} & \begin{minipage}[b]{\linewidth}\raggedright
Method 1
\end{minipage} & \begin{minipage}[b]{\linewidth}\raggedright
Method 2
\end{minipage} & \begin{minipage}[b]{\linewidth}\raggedright
Method 3
\end{minipage} & \begin{minipage}[b]{\linewidth}\raggedright
Best
\end{minipage} \\
\midrule\noalign{}
\endhead
\bottomrule\noalign{}
\endlastfoot
MNIST & 60000 & 10000 & 784 & 10 & 92.3\% & 95.1\% & 96.8\% & 97.2\% &
97.2\% \\
CIFAR-10 & 50000 & 10000 & 3072 & 10 & 75.4\% & 82.3\% & 86.1\% & 89.5\%
& 89.5\% \\
ImageNet & 1281167 & 50000 & 150528 & 1000 & 68.2\% & 74.5\% & 78.9\% &
82.1\% & 82.1\% \\
\end{longtable}

\section{Grouped Tables}\label{sec-grouped}

Organize related results:

\textbf{Table: Results by Category}

\begin{longtable}[]{@{}llll@{}}
\caption{Results grouped by task}\label{tbl-grouped}\tabularnewline
\toprule\noalign{}
Category & Precision & Recall & F1 \\
\midrule\noalign{}
\endfirsthead
\toprule\noalign{}
Category & Precision & Recall & F1 \\
\midrule\noalign{}
\endhead
\bottomrule\noalign{}
\endlastfoot
\textbf{Image Classification} & & & \\
Model A & 85.3\% & 83.1\% & 84.2\% \\
Model B & 89.7\% & 87.5\% & 88.6\% \\
\textbf{Object Detection} & & & \\
Model C & 78.4\% & 76.2\% & 77.3\% \\
Model D & 82.1\% & 80.5\% & 81.3\% \\
\end{longtable}

\section{Statistical Significance}\label{sec-statistics}

\begin{longtable}[]{@{}llll@{}}
\caption{Statistical significance tests (* p\textless0.05, ***
p\textless0.001)}\label{tbl-stats}\tabularnewline
\toprule\noalign{}
Comparison & t-statistic & p-value & Significant \\
\midrule\noalign{}
\endfirsthead
\toprule\noalign{}
Comparison & t-statistic & p-value & Significant \\
\midrule\noalign{}
\endhead
\bottomrule\noalign{}
\endlastfoot
A vs B & 3.45 & 0.001 & Yes*** \\
B vs C & 2.18 & 0.032 & Yes* \\
A vs C & 1.67 & 0.098 & No \\
\end{longtable}

\section{Ablation Study}\label{sec-ablation}

\begin{longtable}[]{@{}lll@{}}
\caption{Ablation study results}\label{tbl-ablation}\tabularnewline
\toprule\noalign{}
Component Removed & Accuracy & Drop \\
\midrule\noalign{}
\endfirsthead
\toprule\noalign{}
Component Removed & Accuracy & Drop \\
\midrule\noalign{}
\endhead
\bottomrule\noalign{}
\endlastfoot
Full Model & 92.3\% & - \\
- Feature A & 89.1\% & -3.2\% \\
- Feature B & 90.5\% & -1.8\% \\
- Feature C & 87.2\% & -5.1\% \\
- All Features & 78.4\% & -13.9\% \\
\end{longtable}

\section{Discussion}\label{sec-discussion}

Analyze and interpret the results:

\textbf{Key Findings:}

\begin{enumerate}
\def\labelenumi{\arabic{enumi}.}
\item
  The proposed method achieves 92.3\% accuracy
  (Table~\ref{tbl-results}), outperforming the baseline by 10\%.
\item
  Statistical tests (Table~\ref{tbl-stats}) confirm the improvements are
  significant (p \textless{} 0.001).
\item
  Ablation study (Table~\ref{tbl-ablation}) shows Feature C is most
  critical (-5.1\% when removed).
\end{enumerate}

\textbf{Comparison with State-of-the-Art:}

As shown in Table~\ref{tbl-aligned}, our Method C achieves the best
performance while maintaining reasonable computational time.

\bookmarksetup{startatroot}

\chapter{Conclusions and Future Work}\label{sec-conclusions}

Summarize your research contributions and outline future directions.

\section{Summary of Work}\label{sec-summary}

Provide a concise summary of your research:

This thesis presented a novel approach to {[}problem domain{]}. The main
contributions include:

\begin{enumerate}
\def\labelenumi{\arabic{enumi}.}
\tightlist
\item
  Development of a new methodology for {[}specific task{]}
\item
  Experimental validation achieving X\% improvement over baseline
\item
  Comprehensive analysis of {[}specific aspect{]}
\end{enumerate}

\section{Key Contributions}\label{sec-contributions}

List your main contributions:

\begin{enumerate}
\def\labelenumi{\arabic{enumi}.}
\item
  \textbf{Contribution 1:} Proposed a new
  {[}method/algorithm/framework{]} that improves {[}metric{]} by X\%
\item
  \textbf{Contribution 2:} Demonstrated that {[}finding{]} leads to
  better {[}outcome{]}
\item
  \textbf{Contribution 3:} Provided comprehensive empirical analysis
  across {[}number{]} datasets
\item
  \textbf{Contribution 4:} Released open-source implementation for
  reproducibility
\end{enumerate}

\section{Limitations}\label{sec-limitations}

Acknowledge limitations of your work:

\begin{itemize}
\tightlist
\item
  The proposed method requires {[}resource/constraint{]}
\item
  Performance degrades when {[}specific condition{]}
\item
  Current implementation is limited to {[}scope{]}
\end{itemize}

\section{Future Research Directions}\label{sec-future}

Suggest areas for future work:

\subsection{Short-term Directions}\label{short-term-directions}

\begin{enumerate}
\def\labelenumi{\arabic{enumi}.}
\tightlist
\item
  Extend the approach to handle {[}new scenario{]}
\item
  Optimize computational efficiency for {[}application{]}
\item
  Investigate performance on {[}additional datasets{]}
\end{enumerate}

\subsection{Long-term Vision}\label{long-term-vision}

\begin{enumerate}
\def\labelenumi{\arabic{enumi}.}
\tightlist
\item
  Develop theoretical foundations for {[}aspect{]}
\item
  Explore integration with {[}related technology{]}
\item
  Apply methodology to {[}broader domain{]}
\end{enumerate}

\section{Closing Remarks}\label{sec-closing}

The findings of this research demonstrate that {[}key takeaway{]}. This
work opens new possibilities for {[}future applications{]}.

\bookmarksetup{startatroot}

\chapter{Publications}\label{sec-publications}

List publications arising from this research work.

\section{Journal Articles}\label{sec-journals}

\subsection{Published}\label{published}

\begin{enumerate}
\def\labelenumi{\arabic{enumi}.}
\item
  \textbf{Author1, A., Author2, B., and Author3, C.} (2024). ``Complete
  Article Title.'' \emph{Journal Name}, vol.~10, no. 2, pp.~123-145.
  DOI:
  \href{https://doi.org/10.1234/journal.2024.001}{10.1234/journal.2024.001}
\item
  \textbf{Author1, A. and Author2, B.} (2023). ``Second Article Title.''
  \emph{Another Journal}, vol.~8, no. 4, pp.~567-589. DOI:
  \href{https://doi.org/10.5678/journal.2023.045}{10.5678/journal.2023.045}
\end{enumerate}

\subsection{Under Review}\label{under-review}

\begin{enumerate}
\def\labelenumi{\arabic{enumi}.}
\setcounter{enumi}{2}
\tightlist
\item
  \textbf{Author1, A., Author3, C., and Author4, D.} ``Title of
  Submitted Paper.'' \emph{Target Journal}, submitted October 2024.
\end{enumerate}

\section{Conference Papers}\label{sec-conferences}

\subsection{Published}\label{published-1}

\begin{enumerate}
\def\labelenumi{\arabic{enumi}.}
\item
  \textbf{Author1, A. and Author2, B.} (2024). ``Conference Paper
  Title.'' In \emph{Proceedings of the International Conference on
  Topic}, pp.~100-110, City, Country. DOI:
  \href{https://doi.org/10.1109/CONF.2024.123456}{10.1109/CONF.2024.123456}
\item
  \textbf{Author1, A., Author3, C., and Author5, E.} (2023). ``Another
  Conference Paper.'' In \emph{IEEE Conference on Related Topic},
  pp.~45-58, City, Country.
\end{enumerate}

\subsection{Accepted}\label{accepted}

\begin{enumerate}
\def\labelenumi{\arabic{enumi}.}
\setcounter{enumi}{2}
\tightlist
\item
  \textbf{Author1, A. and Author2, B.} ``Upcoming Conference Paper.'' In
  \emph{International Conference Name}, to appear, 2025.
\end{enumerate}

\section{Book Chapters}\label{sec-chapters}

\begin{enumerate}
\def\labelenumi{\arabic{enumi}.}
\tightlist
\item
  \textbf{Author1, A. and Author2, B.} (2024). ``Chapter Title.'' In
  \emph{Book Title}, Editor, Ed. Publisher, pp.~78-102. ISBN:
  978-0-123456-78-9
\end{enumerate}

\section{Preprints and Technical Reports}\label{sec-preprints}

\begin{enumerate}
\def\labelenumi{\arabic{enumi}.}
\tightlist
\item
  \textbf{Author1, A., Author2, B., and Author3, C.} ``Preprint Title.''
  arXiv:2024.12345, 2024. Available:
  \url{https://arxiv.org/abs/2024.12345}
\end{enumerate}

\section{Patents}\label{sec-patents}

\begin{enumerate}
\def\labelenumi{\arabic{enumi}.}
\tightlist
\item
  \textbf{Author1, A. and Author2, B.} ``Patent Title.'' Patent
  No.~US1234567B2, filed Jan 2023, granted Dec 2024.
\end{enumerate}

\section{Publication Metrics}\label{sec-metrics}

\begin{longtable}[]{@{}ll@{}}
\caption{Publication summary}\label{tbl-pub-metrics}\tabularnewline
\toprule\noalign{}
Type & Count \\
\midrule\noalign{}
\endfirsthead
\toprule\noalign{}
Type & Count \\
\midrule\noalign{}
\endhead
\bottomrule\noalign{}
\endlastfoot
Journal Articles & 2 \\
Conference Papers & 2 \\
Under Review & 1 \\
Citations (Total) & 45 \\
h-index & 3 \\
\end{longtable}

\section{Awards and Recognition}\label{sec-awards}

\begin{itemize}
\tightlist
\item
  \textbf{Best Paper Award:} Conference Name, 2024
\item
  \textbf{Outstanding Reviewer:} Journal Name, 2023
\end{itemize}

\bookmarksetup{startatroot}

\chapter*{References}\label{references}
\addcontentsline{toc}{chapter}{References}

\markboth{References}{References}

All cited references will appear here automatically.

\printbibliography[heading=none]





\end{document}
